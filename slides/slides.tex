\documentclass{beamer}

\setbeamertemplate{note page}[plain]

\usetheme{metropolis}

\setbeamercolor{background canvas}{bg=white}

\usefonttheme{serif}

\usepackage{setspace}
\usepackage{longtable}
\usepackage{booktabs}
\usepackage{graphicx}
\usepackage[rightcaption]{sidecap}
\usepackage{mathtools}
\usepackage{amsmath}
\usepackage{dsfont}

\usepackage{xcolor}

\usepackage{caption}
\usepackage{subcaption}
\captionsetup{font = tiny}

\usepackage{setspace}
\usepackage{array}
\usepackage{arydshln}

\usepackage{natbib}
\bibliographystyle{unsrtnat}

\usepackage{catchfile}

\usepackage{multicol}

\newtheorem{property}{Property}

\hfuzz = 60pt
\hfuzz = -60pt

\usepackage{enumitem}
\setlist[enumerate]{font=\bfseries, label=(\alph*), itemsep=0.5cm}


\title{\fontsize{12pt}{15pt} \selectfont HOURLY PRICE FORWARD CURVE}

\author{Dr. Gaygysyz Guljanov}

\institute{University of Münster}

\date{04.11.2024}
% -----------------------------------------------------------------


% === Document begins ===
\begin{document}

\frame{\titlepage}

% === Outline ===
\section{Outline}

\begin{frame}{Outline}
    \tableofcontents
\end{frame}


% === Data Pre-processing ===
\section{Data Pre-processing}

\begin{frame}{Missing values \& Outliers}
    \begin{enumerate}
        \item \textbf{Missing Values:} Following \cite{SvantessonRastegar-2019}, I use linear interpolation to fill in the missing values.

        \item \textbf{Outliers:} Following \cite{SvantessonRastegar-2019}, I use 99 percentile and 168 day moving average
    \end{enumerate}
\end{frame}


\begin{frame}{Raw Data}
    \includegraphics[width=0.95\textwidth]{../Figs/fig_rawdata.pdf}
\end{frame}


\begin{frame}{Raw Data with Outliers}
    \includegraphics[width=0.95\textwidth]{../Figs/fig_outliers.pdf}
\end{frame}


\begin{frame}{Raw Data without Outliers}
    \includegraphics[width=0.95\textwidth]{../Figs/fig_wo_outliers.pdf}
\end{frame}



% === Theoretical Background ===
\section{Theoretical Background}

\begin{frame}{Theoretical Background -- \cite{BurgerGraeberSchindlmayr-2014}}
    The valuation using the hourly forward curve must match the valuation using market quotes from exchanges:
    \[
        \frac{1}{|J^b| \sum_{i \in J^b} F(t, T_i)} = F^b
    \]
    \begin{enumerate}[label=--, itemsep=0.1cm]
        \item $F_b$ is the price of a standard baseload contract (e.g. a certain calendar year)

        \item $J^b$ is the set of hours within the delivery year

        \item $F(t, T_i)$ is a forward curve at current time $t$ for all delivery hours (or days) $T_i (i = 1, \ldots, N)$
    \end{enumerate}
\end{frame}


\begin{frame}{Game Plan -- \cite{BurgerGraeberSchindlmayr-2014}}
    \begin{enumerate}
        \item[Step 1:] Model the weekly and yearly shape of average daily prices, that is calculate daily shape factors from historical spot prices

        \item[Step 2:] Model the (hourly) intra-day shape for different profile types depending on weekday and season (not required for daily forward curves)

        \item[Step 3:] Apply the yearly and intra-day shape factors to quoted forward prices
    \end{enumerate}
\end{frame}


\begin{frame}{Game Plan Details -- \cite{BurgerGraeberSchindlmayr-2014}}
    \begin{enumerate}[label=$\bullet$, itemsep=0.1cm]
        \item Normalized daily spot prices, i.e. baseload spot price at day $t$ divided by the average spot price of that year:
              \[ y(t) \]

        \item Normalized spot price for the $k$th hour of the day $t$:
              \[ y_k(t), \quad k = 1, \ldots, 24 \]

        \item Normalized hourly spot shape:
              \[
                  h_k(t)
                  = \frac{y_k(t)}{y(t)}
                  = \frac{\text{Hourly Price}}{\text{Daily Price}}
              \]
    \end{enumerate}
\end{frame}


\begin{frame}{Game Plan Details -- \cite{BurgerGraeberSchindlmayr-2014}}
    \begin{enumerate}[label=$\bullet$, itemsep=0.1cm]
        \item Dummy variables to model seasonalities
              \[
                  y(t)
                  = \sum_{j = 1}^N \beta_j \mathds{1}_{D_j}(t) + \epsilon(t)
              \]
              \begin{enumerate}
                  \item $D_j$: set of all days within cluster $j$
              \end{enumerate}

        \item Approach can be refined by using a more complex regression model (e.g. temperature as an explanatory variable)
    \end{enumerate}
\end{frame}


\begin{frame}{Game Plan Details -- \cite{BurgerGraeberSchindlmayr-2014}}
    \begin{enumerate}[label=--]
        \item Hourly Forward Curve for the $k$th hour of day $t$ (denoted by $T$) can be defined as:
              \[
                  F(t_0, T) = \hat y(t) \hat h_k(t) F
              \]
              \begin{enumerate}[label=$\bullet$]
                  \item $F$: a yearly baseload forward price
              \end{enumerate}

        \item Above, the shape factors are applied in such a way that the resulting hourly forward curve is consistent with quoted forward prices.
    \end{enumerate}
\end{frame}


\begin{frame}{Normalized Hourly Prices}
    \includegraphics[width=0.95\textwidth]{../Figs/fig_nh_preis.pdf}
\end{frame}


\begin{frame}{Normalized Daily Prices}
    \includegraphics[width=0.95\textwidth]{../Figs/fig_nd_preis.pdf}
\end{frame}


% \begin{frame}{Non-overlapping contracts -- \cite{BurgerGraeberSchindlmayr-2014}}
%     Non-overlapping contracts:
%     \begin{enumerate}
%         \item If there are several non-overlapping market forward prices, the shape factors can be normalized to mean $1$ during each contract period and the same method can be used.

%               \begin{enumerate}
%                   \item The shape factors $\hat h_k(t)$ can be normalized to mean $1$ separately for peak and off-peak hours and applied to the corresponding peak and off-peak forward price.

%                   \item As the peak and off-peak hours are scaled by different shape factors in this case, a discontinuity may arise at the intersection betweek peak and off-peak. This might require an additional smoothing procedure.
%               \end{enumerate}
%     \end{enumerate}
% \end{frame}



% === Seasonality clusters ===
\section{Seasonality clusters}

\begin{frame}{Seasonalities within a Year}
    We should choose the number of clusters carefully, to have sufficient historical data.

    Following \cite{SteinErikParaschivSchuerle-2013}, we use the following dummy variables to model the seasonalities for daily prices:
    \begin{enumerate}[label=--]
        \item $6$ dummy variables (Mo-Sa)

        \item $11$ monthly dummy variables (Feb-Dec)

        \item August should be subdivided in two parts, due to summer vacations
    \end{enumerate}
\end{frame}


\begin{frame}{Box Plots per Month}
    \includegraphics[width=0.95\textwidth]{../Figs/fig_box_month.pdf}
\end{frame}


\begin{frame}{Box Plots per Weekday}
    \includegraphics[width=0.95\textwidth]{../Figs/fig_box_day.pdf}
\end{frame}



\begin{frame}{Seasonalities within a Day}
    Following \cite{SteinErikParaschivSchuerle-2013}, we use the following dummy variables to model the seasonalities for hourly prices:

    \footnotesize
    \begin{tabular}{l|cccccccccccc}
         & J  & F  & M  & A  & M  & J  & J  & A  & S  & O  & N  & D
        \\
        \toprule
        Week day
         & 1  & 2  & 3  & 4  & 5  & 6  & 7  & 8  & 9  & 10 & 11 & 12
        \\
        Sat
         & 13 & 13 & 14 & 14 & 14 & 15 & 15 & 15 & 16 & 16 & 16 & 13
        \\
        Sun
         & 17 & 17 & 18 & 18 & 18 & 19 & 19 & 19 & 20 & 20 & 20 & 17
    \end{tabular}

    \vspace*{0.8cm}
    The above points could contain holidays and bridge days.
\end{frame}


\begin{frame}{Box Plots for Seasonalities within a Day}
    Figures are coming
\end{frame}


% === Assumptions ===
\section{Assumptions}

\begin{frame}{Assumptions}
    \begin{enumerate}[label=\roman*.]
        \item Linearity
        \item Symmetry
        \item Discreteness arising from dummy variables
        \item ...
    \end{enumerate}
\end{frame}


% === Estimation Results ===
\section{Estimation Results}

\begin{frame}{Estimation Results}
    The Results are coming soon. For now, it can be accessed with running Python Codes.
\end{frame}


\begin{frame}{Actual vs. Predict values}
    \includegraphics[width=0.95\textwidth]{../Figs/fig_h_pred.pdf}
\end{frame}


\begin{frame}{Backtesting}
    \includegraphics[width=0.95\textwidth]{../Figs/fig_h_pred_out.pdf}
\end{frame}



\begin{frame}{Autocorrelations before and after Seasonalities}
    The plot is coming soon
\end{frame}


% === Refinements and Improvements ===
\section{Refinements and Improvements}

\begin{frame}{Possible Improvements}
    \begin{enumerate}[label=(\roman*)]
        \item Modeling the error term:
              Extra Regressors: Temperature of the day, CDD and HDD,

        \item Other models and functional forms

        \item The combination of the two
    \end{enumerate}
\end{frame}


\begin{frame}{Possible Improvements}
    Extra Regressors
    \begin{enumerate}[label=(\roman*)]
        \item Temperature of the day

        \item CDD and HDD,

        \item
    \end{enumerate}
\end{frame}


\begin{frame}{Modelling the Error Term}
    Polynomial Regression
\end{frame}


\begin{frame}{Modelling the Error Term}
    Using Structural trends: from \cite{deJongDijkenEnev-2013} and \cite{BurgerGraeberSchindlmayr-2014}
    \[
        \Delta s(t) = -\beta_1 s(t-24)
        - \beta_2 \Delta p_{\text{solar}}(t)
        - \beta_3 \Delta p_{\text{wind}}(t)
        + \epsilon(t)
    \]
    \begin{enumerate}[label=$\bullet$]
        \item $s(t)$: spot price for hour $t$
        \item $\Delta$: lag operator with period $24$
        \item $p_{\text{solar}}(t)$: wind production
        \item $p_{\text{wind}}(t)$: solar production
    \end{enumerate}
\end{frame}



\begin{frame}{Other models and functional forms}
    Bid-offer prices \& Smoothing from \cite{FletenLemming-2003} and \cite{BurgerGraeberSchindlmayr-2014}:
    \[
        \min \sum_{t=1}^T \bigg(f(t) - s(t) \bigg)^2
        + \lambda \sum_{t=2}^{T-1} \bigg(
        f(t-1) - 2 f(t) + f(t+1)
        \bigg)^2
    \]
    subject to
    \begin{align*}
        F(T_{1i}, T_{2i})_{bid}
         & \leq
        \frac{1}{\sum_{t=T_{1i}}^{T_{2i}}} e^{-rt}
        \sum_{t=T_{1i}}^{T_{2i}} e^{-rt} f(t);
        \quad \forall i \in \mathcal{F}
        \\
        F(T_{1i}, T_{2i})_{offer}
         & \geq
        \frac{1}{\sum_{t=T_{1i}}^{T_{2i}}} e^{-rt}
        \sum_{t=T_{1i}}^{T_{2i}} e^{-rt} f(t);
        \quad \forall i \in \mathcal{F}
    \end{align*}
\end{frame}


\begin{frame}{Other models and functional forms}
    Fourier or trigonometric functions
\end{frame}


% === The End ===
\section{The End}

\begin{frame}
    \centering
    \Huge
    Thank you for your attention! \\[0.5cm] Questions \& Discussions?
\end{frame}

% Put the bibliography
\nocite{*}
\bibliography{biblio.bib}

\end{document}
