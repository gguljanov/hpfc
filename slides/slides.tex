\documentclass{beamer}

\setbeamertemplate{note page}[plain]

\usetheme{metropolis}

\setbeamercolor{background canvas}{bg=white}

\usefonttheme{serif}

\usepackage{setspace}
\usepackage{longtable}
\usepackage{booktabs}
\usepackage{graphicx}
\usepackage[rightcaption]{sidecap}
\usepackage{mathtools}
\usepackage{amsmath}
\usepackage{dsfont}

\usepackage{xcolor}

\usepackage{caption}
\usepackage{subcaption}
\captionsetup{font = tiny}

\usepackage{setspace}
\usepackage{array}
\usepackage{arydshln}

\usepackage{natbib}
\bibliographystyle{unsrtnat}

\usepackage{catchfile}

\usepackage{multicol}

\newtheorem{property}{Property}

\hfuzz = 60pt
\hfuzz = -60pt

\title{\fontsize{12pt}{15pt} \selectfont HOURLY PRICE FORWARD CURVE}

\author{Dr. Gaygysyz Guljanov}

\institute{University of Münster}

\date{04.11.2024}



\begin{document}

\frame{\titlepage}


% === Data Pre-processing ===
\section{Data Pre-processing}

\begin{frame}{Missing values}
    Following \cite{SvantessonRastegar-2019}, I use linear interpolation to fill in the missing values.
\end{frame}


\begin{frame}{Outliers}
    \begin{itemize}
        \item Following \cite{SvantessonRastegar-2019}, I use 99 percentile

        \item 168 day moving average, to fill in
    \end{itemize}
\end{frame}



% === Theoretical Background ===
\section{Theoretical Background}

\begin{frame}
    \begin{itemize}
        \item The valuation using the hourly forward curve must match the valuation using market quotes from exchanges.

        \item
              \[
                  \frac{1}{|J^b| \sum_{i \in J^b} F(t, T_i)} = F^b
              \]
              \begin{itemize}
                  \item $F_b$ is the price of a standard baseload contract (e.g. a certain calendar year)
                  \item $J^b$ is the set of hours within the delivery year
                  \item $F(t, T_i)$ is a forward curve at current time $t$ for all delivery hours (or days) $T_i (i = 1, \ldots, N)$
              \end{itemize}
    \end{itemize}
\end{frame}


\begin{frame}
    \begin{itemize}
        \item[Step 1:] Model the weekly and yearly shape of average daily prices, that is calculate daily shape factors from historical spot prices

        \item[Step 2:] Model the (hourly) intra-day shape for different profile types depending on weekday and season (not required for daily forward curves)

        \item[Step 3:] Apply the yearly and intra-day shape factors to quoted forward prices
    \end{itemize}
\end{frame}


\begin{frame}
    \begin{itemize}
        \item $y_t$: normalized daily spot prices, i.e. baseload spot price at day $t$ divided by the average spot price of that year

        \item $y_k(t)$: normalized spot price for the $k$th hour of the day $t (k = 1, \ldots, 24)$

        \item $h_k(t) = y_k(t) / y(t)$: normalized hourly spot shape

        \item Dummy variables to model seasonalities
              \[
                  y(t)
                  = \sum_{j = 1}^N \beta_j \mathds{1}_{D_j}(t) + \epsilon(t)
              \]
              \begin{itemize}
                  \item $D_j$: set of all days within cluster $j$
              \end{itemize}

        \item Treat outliers carefully

        \item Approach can be refined by using a more complex regression model (e.g. temperature as an explanatory variable)
    \end{itemize}
\end{frame}


\begin{frame}
    \begin{itemize}
        \item Hourly Forward Curve for the $k$th hour of day $t$ (denoted by $T$) can be defined as:
              \[
                  F(t_0, T) = \hat y(t) \hat h_k(t) F
              \]
              \begin{itemize}
                  \item $F$: a yearly baseload forward price
              \end{itemize}

        \item Above, the shape factors are applied in such a way that the resulting hourly forward curve is consistent with quoted forward prices.
    \end{itemize}
\end{frame}


\begin{frame}
    Non-overlapping contracts:
    \begin{itemize}
        \item If there are several non-overlapping market forward prices, the shape factors can be normalized to mean $1$ during each contract period and the same method can be used.

              \begin{itemize}
                  \item The shape factors $\hat h_k(t)$ can be normalized to mean $1$ separately for peak and off-peak hours and applied to the corresponding peak and off-peak forward price.

                  \item As the peak and off-peak hours are scaled by different shape factors in this case, a discontinuity may arise at the intersection betweek peak and off-peak. This might require an additional smoothing procedure.
              \end{itemize}
    \end{itemize}
\end{frame}


\begin{frame}
    Overlapping contracts: ...
\end{frame}



% === Seasonality clusters ===
\section{Seasonality clusters}

\begin{frame}
    Possible clusters
    \begin{itemize}
        \item Choose number of clusters carefully, to have sufficient historical data

        \item (i) Monday, (ii) Tu-Th, (iii) Fr, (iv) Sa, (v) Su

        \item The above point could contain holidays and bridge days
    \end{itemize}
\end{frame}


\begin{frame}
    Refinements:
    \begin{itemize}
        \item Off-standard contracts
        \item Bid-offer prices and smoothing
        \item Structural trends
    \end{itemize}
\end{frame}


% Put the bibliography
\nocite{*}
\bibliography{biblio.bib}

\end{document}
